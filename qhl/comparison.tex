\section{Related Work}
\label{sec:related}

\myparagraph{Quantum Proof Systems and Verification Frameworks}
Previous quantum proof systems, including quantum Hoare Logics \cite{qhoare,qhoreusage,10.1145/3456877,10.1007/s00165-018-0465-3},
quantum separation logics \cite{qseplocal,qsepa}, and quantum relational logics \cite{relationlogic,10.1145/3290346}, enlightened the development of QNP. The problems of these works are three: 1) their conditionals are solely classical, while QNP has quantum conditionals; 2) most of them are theoredical works or implemented as tactics in an interactive theorem provers, so that it is unclear if they can be implemented in a classical computer and utilize classical SMT solvers for proof automation; and 3) they did not provide models for compiling quantum programs to circuits. 
Kakutani \cite{10.1007/978-3-642-10622-4_7} provided a quantum logic by extending the probabilistic Hoare logic \cite{10.1007/3-540-46674-6_11} with quantum conditionals, but the proof rule is a semantic interpretation of quantum controls that requires the evaluation of the whole quantum state when proving quantum conditional properties. It is more of a symbolic semantic framework than a proof system. Quantum separation logic \cite{qseplocal} discusses a frame rule that indicates a $\thadt$ type quantum state can be split into two parts, which is similar to our $\thadt$ type state split equations but different from the frame rule in QNP based directly on classical separation logic frame rule and utilizing the \qafny type system to ensure the separation. 

There are works on formally verifying quantum programs includes Qwire~\cite{RandThesis}, SQIR~\cite{PQPC}, and QBricks~\cite{qbricks}. These tools have been used to verify a range of quantum algorithms, from Grover's search to quantum phase estimation.
The former two tools provided libraries in a proof assistant to help verify quantum programs and they have circuit compilation models, while the latter one built a proof system on top of a proof assistant to achieve some proof automations without providing a circuit compilation model. The system comparision of QBricks and QNP is given in \Cref{sec:arith-oqasm}.

\myparagraph{Classical Proof Systems}
The \qafny proof system is enlightened by the classical seapration logic \cite{separationlogic}, and many others \cite{10.1145/3453483.3454087,arxiv.1609.00919,10.1007/978-3-319-89960-2_13,10.1007/978-3-319-89960-2_2,nat-proof-fun}. Especially, the \qafny proof system is compiled to Dafny \cite{10.1007/978-3-642-17511-4_20}, a mechanized separation logic system. The methodology of QNP is inheriented from the natural proof methodology \cite{nat-proof-fun,nat-proof-frame,10.1145/2103621.2103673}, which is discussed in \Cref{sec:background}.



