\section{Background}
\label{sec:background}

Here, we provide background information for quantum computing with an example in \qafny.

\myparagraph{Quantum States} A quantum state consists of one or more quantum bits (\emph{qubits}). A qubit can be expressed as a two dimensional vector $\begin{psmallmatrix} \alpha \\ \beta \end{psmallmatrix}$ where $\alpha,\beta$ are complex numbers such that $|\alpha|^2 + |\beta|^2 = 1$.  The $\alpha$ and $\beta$ are called \emph{amplitudes}. 
%
We frequently write the qubit vector as $\alpha\ket{0} + \beta\ket{1}$ where $\ket{0} = \begin{psmallmatrix} 1 \\ 0 \end{psmallmatrix}$ and $\ket{1} = \begin{psmallmatrix} 0 \\ 1 \end{psmallmatrix}$ are \emph{computational basis states}. When both $\alpha$ and $\beta$ are non-zero, we can think of the qubit as being ``both 0 and 1 at once,'' a.k.a. a \emph{superposition}. For example, $\frac{1}{\sqrt{2}}(\ket{0} + \ket{1})$ is an equal superposition of $\ket{0}$ and $\ket{1}$. 
We can join multiple qubits together to form a larger quantum state with the \emph{tensor product} ($\otimes$) from linear algebra. For example, the two-qubit state $\ket{0} \otimes \ket{1}$ (also written as $\ket{01}$) corresponds to vector $[~0~1~0~0~]^T$. 
Sometimes a multi-qubit state cannot be expressed as the tensor of individual states; such states are called \emph{entangled}. One example is the state $\frac{1}{\sqrt{2}}(\ket{00} + \ket{11})$, known as a \emph{Bell pair}.

\myparagraph{Alternative State Representation} $n$-qubit quantum states are typically represented as $2^n$ dimensional vectors above. Alternatively, the state can be represented as different forms. For example, a newly generated qubit typically has a state $\ket{0}$ or $\ket{1}$, which is named \textit{normal typed state} ($\tnort$) in QNP. Qubits that are in superposition but not entangled, such as $\frac{1}{\sqrt{2}}(\ket{0} + \ket{1})\otimes \frac{1}{\sqrt{2}}(\ket{0} + \ket{1})$, can be expressed as a summation of tensor products as $\shad{2^n}{n}{\alpha(r_j)}$, where $\alpha(r_j)=e^{2\pi i r}$ and $r \in \mathbb{R}$, which is named \textit{Hadamard typed state} ($\thadt$) in QNP. The above two qubit superposition can be expressed as $\shad{4}{2}{}$.
$\alpha(r_j)$ is named the \emph{local phase} of the state, which are special quantum amplitudes (see below) such that the norm is $1$, i.e., $\slen{\alpha(r_j)}^2=1$. In the above state, we can view the local phase $1$ as $e^{0}$, and $\frac{1}{\sqrt{4}}e^{0}$ is the amplitude for every basis state.

The most general representation is to express quantum states as a path-sum formula as: $\sch{m}{z_j}{c_j}$, where $z_j\in \mathbb{C}$ is named \emph{amplitude}, $c_j$ is an $n$-length bitstring named \emph{basis}, and $m \le 2^n$. Each $j$-th term ${z_j}\ket{c_j}$ in the formula represents a \emph{basis state} in a superposition state as ${z_0}\ket{c_0}+...+{z_{m-1}}\ket{c_{m-1}}$. 
This is named \textit{entanglement typed state} ($\tcht$) in QNP.
For example, the bell pair can be represented as $\sch{2}{\frac{1}{\sqrt{2}}}{c_j}$ with $c_0=00$ and $c_1=11$.
Notice that the amplitudes satisfy the relation $\sum_{0}^{m}\slen{z_j}^2 = 1$. However, in some intermediate program evaluation in QNP, we loose the restriction to be $\sum_{0}^{m}\slen{z_j}^2 \le 1$, because the state $\sch{m}{z_j}{c_j}$ can be split into two parts as $\sch{m}{z_j}{c_j}=\schii{m_1}{z_i}{c_i}+\schk{m_2}{z_k}{c_k}$, and we might only want to reason about a portion of the state $\sch{m_1}{z_j}{c_j}$ locally, so that $\sum_{0}^{m_1}\slen{z_i}^2 < 1$. Obviously, in the top-most program evaluation level, every state satisfies the restriction that $\sum_{0}^{m}\slen{z_j}^2 = 1$.
\ignore{
\myparagraph{Sessions}
If we analogize a quantum state $\sch{m}{z_j}{c_j}$ as an $m$-length array and each basis state $z_j\ket{c_j}$ as the $j$-th element in the array. \Cref{fig:background-circuit-example-proof} line 1, we have a singleton array $x[0..n]$, each element of which is $n$-length.
In QNP, we named the name structure referring to an array representing a quantum state as a \emph{session}. Here $x[0..n]$ is a session indicating that the location of the array is in a region named $x$, and the range is from $0$ to $n$, which means that the length of each element is $n$. After the application of $\ssassign{x[0]}{}{\ihadh}$, we split the session into two parts: $x[0..1]$ and $x[1..n]$ having a $\thadt$ type ($\frac{1}{\sqrt{2}}(\ket{0}+\ket{1})$) and $\tnort$ ($\ket{\overline{0}}$) type states, respectively.
Here, the singleton element in $x[1..n]$ is now $n-1$ length.
State $\frac{1}{\sqrt{2}}(\ket{0}+\ket{1})$ can also be viewed as $\sch{2}{\frac{1}{\sqrt{2}}}{\overline{i}}$ with $i\in[0,1]$.
Here, $\overline{i}$ is a singleton bitstring. 
In each step of the loop in line 4-6, we cut off the $j$-th bit in the singleton element array, so the array session becomes $x[\snext{j}..n]$, then we put the bit in the original $x[0..1]$ session which is now structured as $x[0..j]$.
The state becomes $\sch{2}{\frac{1}{\sqrt{2}}}{\overline{i}}\ket{0}$. Notice that the quantum conditional in each loop step applies on the $j$-th and $(\snext{j})$-th bits of the two basis states ($\overline{0}$ and $\overline{1}$). When $i=0$, we do nothing; if $i=1$, we flip the bit, which means that we turn the state to be $\sch{2}{\frac{1}{\sqrt{2}}}{\overline{i}}$, with the $\overline{i}$ length increases by one and it is the same as the length the session $x[0..\snext{j}]$. In the end, we can remove the $x[n..n]\mapsto \ket{\overline{0}}$ in the final state, because session $x[n..n]$ means that each element in the state array is $0$, so we claim such session as empty. 
}

\myparagraph{Quantum Computations and Conditionals} In the \emph{QRAM model}~\cite{Knill1996} quantum computers are used as co-processors to classical computers. The classical computer generates descriptions of circuits to send to the quantum computer and then processes the measurement results. High-level quantum programming languages are designed to follow this model.
Computation on a quantum state consists of a series of \emph{quantum operations}, each of which acts on a subset of qubits in the quantum state. In the standard presentation, quantum computations are expressed as \emph{circuits}, as shown in \Cref{fig:background-circuit-examplea}, which constructs a circuit that prepares the Greenberger-Horne-Zeilinger (GHZ) state \cite{Greenberger1989}, which is an $n$-qubit entangled quantum state of the form:
\begin{align*}
    \ket{\text{GHZ}^n} = \frac{1}{\sqrt{2}}(\ket{0}^{\otimes n}+\ket{1}^{\otimes n}).
\end{align*}

In these circuits, each horizontal wire represents a qubit and boxes on these wires indicate quantum operations, or \emph{gates}. The circuit in \Cref{fig:background-circuit-examplea} uses $n$ qubits and applies $n$ gates: a \emph{Hadamard} (\coqe{H}) gate and $n-1$ \emph{controlled-not} (\coqe{CNOT}) gates.
Applying a gate to a state \emph{evolves} the state. 
The \qafny implementation in \Cref{fig:background-circuit-example-proof} shows the evolving. In the $j$-th loop step, the quantum state of array $x[0..j]$ is $\frac{1}{\sqrt{2}}(\ket{\overline{0}}+\ket{\overline{1}})$ \footnote{$\overline{0}$ and $\overline{1}$ have the same length as $x[0..j]$}, and a qubit $\ket{0}$ is added to the state and transforms it to $\frac{1}{\sqrt{2}}(\ket{\overline{0}}\ket{0}+\ket{\overline{1}}\ket{0})$, which adds a bit $0$ to the end of every basis state. Then, the quantum conditional 
($\sifq{x[j]}{\ssassign{x[\snext{j}]}{}{x[\snext{j}]+1}}$) turns $\ket{0}$ in the second basis state to $\ket{1}$ as $\frac{1}{\sqrt{2}}(\ket{\overline{0}}\ket{0}+\ket{\overline{1}}\ket{1})$, because quantum conditionals checks every the $j$-th qubit in every basis state, if it is $1$, then the conditional body is applied; otherwise, it does nothing.

\myparagraph{Measurement} A special, non-unitary \emph{measurement} operation is used to extract classical information from a quantum state, typically when a computation completes. Measurement collapses the state to one of the basis states with a probability related to the state's amplitudes. For example, measuring $\frac{1}{\sqrt{2}}(\ket{0} + \ket{1})$ will collapse the state to $\ket{0}$ with probability $\frac{1}{2}$ and likewise for $\ket{1}$, returning classical values 0 or 1, respectively.

\myparagraph{Quantum Oracles} Quantum algorithms manipulate input information encoded in ``oracles,'' which are callable black box circuits. For example, Grover's algorithm for unstructured quantum search \cite{grover1996,grover1997} is a general approach for searching a quantum ``database,'' which is encoded in an oracle for a function $f : \{0, 1\}^n \to \{0, 1\}$. Grover's finds an element $x \in \{0, 1\}^n$ such that $f(x) = 1$ using $O(2^{n/2})$ queries, a quadratic speedup over the best possible classical algorithm, which requires $\Omega(2^n)$ queries. Oracles are typically viewed as quantum reversible implementations of classical operations, especially arithmetic operations. OQASM \cite{oracleoopsla} is a language that permits the effective testing of quantum oracles.

\myparagraph{Dafny and Natural Proof} Dafny \cite{10.1007/978-3-642-17511-4_20} is a language that is designed to make it easy to write correct code. It permits imperative programming with logical specifications which can be automatically verified through the Dafny proof system, a separation logic based system. The natural proof methodology was first proposed by Madhusudan \textsf{et al.} \cite{nat-proof,10.1145/2103621.2103673}, which exploits a fixed set of proof tactics, keeping the expressiveness of powerful logics, retaining the
automated nature of proving validity, but giving up on completeness, e.g., the \qafny to Dafny compilation is only sound but not complete. 
The \qafny implementation of the natural proof methodology identifies a subclass of proofs $\mathpzc{N}$ such that
(1) a large class of valid verification specifications of near term classical quantum hybrid programs have a proof in $\mathpzc{N}$, 
and (2) searching for a proof in $\mathpzc{N}$ is efficiently decidable. 
In the original natural proof, the subclass identification is based on mapping heap data-structures, such as trees and link lists, to logical invariant properties. In QNP, the identification is through the \qafny type system in classifying quantum sessions and state types so that quantum operation applications on a specific session can be compiled to classical aggregate operations that have rich proof infrastructures. 










